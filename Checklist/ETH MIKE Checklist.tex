% Template article for preprint document class `elsart'
% with harvard style bibliographic references
% SP 2001/01/05

\documentclass[11pt,a4paper]{paper}

\usepackage{graphicx}
\usepackage{amssymb}
\usepackage{placeins}
\usepackage{float}

\usepackage{enumitem}
\newlist{todolist}{itemize}{2}
\setlist[todolist]{label=$\square$}

\usepackage{hyperref}	%has to be the last called!
\hypersetup{pdfborder={0 0 0},
	pdfauthor={Mike Domenik Rinderknecht, Monika Zbytniewska-Mégret, Manuel Yves Galliker},
	colorlinks=true,
	linkcolor=black,
	citecolor=black,
	urlcolor=black}




\addtolength{\oddsidemargin}{-2.5cm}
\addtolength{\topmargin}{-1cm}
\setlength{\textwidth}{7in}
\addtolength{\textheight}{3cm}

\setlength{\parindent}{0in}

\usepackage{calc}
\newcommand{\figurewidth}{\columnwidth*\real{0.75}}

%=====================================================================================
   
\begin{document}

\title{\vspace{-1cm}ETH MIKE -- Checklist v2.0}

\author{Monika Zbytniewska-Mégret}
\institution{
	Rehabilitation Engineering Laboratory\\
	ETH Zurich\\
	BAA, Lengghalde 5\\
	8008 Zurich\\
	Switzerland}

\maketitle

\vfill

\textbf{At initial startup}
\begin{todolist}
	\setlength\itemsep{-0.25em}
	\item Plug in robot and the tablet charger
	\item Connect emergency button
	\item Switch on tablet
	\item Ensure emergency button is released
	\item Ensure the end-effector is locked in the middle of the robot's workspace
	\item Switch on the robot
	\item Wait until a "click" sound, which indicates that calibration is performed. \\ After the sound you can move the end-effector freely 
	\item Login on the tablet
	\item Start application \texttt{ETH MIKE} on the desktop
	\item Check if the communication symbol is not shown in red color
\end{todolist}

\textbf{After initial startup and if emergency button is pressed:}
\begin{todolist}
	\setlength\itemsep{-0.25em}
	\item Release emergency button
	\item Reset the robot and the tablet application 
\end{todolist}

\textbf{Before running the experiments}
\begin{todolist}
	\setlength\itemsep{-0.25em}
	\item Disinfect hands with hand sanitizer
	\item Ensure that the patient sits in an upright position in front of the device
	\item Put on the bandage and strap and wrist splint to the patients wrist
	\item Add pillows for the arm and back as necessary (see reference in Figure \ref{fig:MIKE_setup})
	\item Put patient's hand on the handle and ensure that the patient holds it
	\item Place the tablet on the mount of the robot, above the hand
\end{todolist}

\textbf{During the experiments}
\begin{todolist}
	\setlength\itemsep{-0.25em}
	\item Login with subject ID code or create new one (button \texttt{Subject Login})
	\item Start with the desired task (depending on the study protocol)
	\item Ensure that the patient sits in an upright position in front of the device
	\item Remove the tablet to ensure that the patients hand is properly positioned and holding on to the handle after the completion of each task
\end{todolist}

\textbf{After the experiments}
\begin{todolist}
	\setlength\itemsep{-0.25em}
	\item Disinfect the handles of the device
\end{todolist}

\begin{figure}[H]
    \centering
    \includegraphics{images/setup_ETHMIKE_example.png}
    \caption{Reference for patient posture and pillow position }
    \label{fig:MIKE_setup}
\end{figure}

%=====================================================================================
\vspace{1cm}

\section*{Contact information}
\label{s:contact_information}

\noindent
Rehabilitation Engineering Laboratory\\
Monika Zbytniewska-Mégret\\
ETH Zurich\\
BAA, Lengghalde 5\\
8008 Zurich\\
+41779615277\\
monika.zbytniewska@hest.ethz.ch\\
http://www.relab.ethz.ch\\

\noindent
Rehabilitation Engineering Laboratory\\
Olivier Lambercy\\
ETH Zurich\\
BAA, Lengghalde 5\\
8008 Zurich\\
+41445107246\\
olivier.lambercy@hest.ethz.ch\\
http://www.relab.ethz.ch\\

\end{document}